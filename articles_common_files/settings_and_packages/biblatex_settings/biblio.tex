%%%%% BIBLIOGRAPHY
%% Troubleshooting: check README
\usepackage{csquotes}

\ifthenelse{\boolean{isIncludeCitationsInFootnotes}}
{%
    \newcommand{\myCiteStyle}{authoryear}%
    \newcommand{\myBibStyle}{authoryear}%
}
{%
    \newcommand{\myCiteStyle}{numeric}%
    \newcommand{\myBibStyle}{numeric}%
}



\usepackage[
    %sorting=none,
    sorting=nty, %name, title, year
    %%% CITATION STYLES: "style applies to both citations in text and display in printed bibliography, citestyle and bibstyles splits
    % Style option examples: numeric, authoryear, authortitle, verbose
    % style=authoryear,
    citestyle=\myCiteStyle,
    bibstyle=\myBibStyle,
    backend=biber,
    datamodel=mydatamodels,
    doi=false,
    isbn=false,
    url=false,
    eprint=false
    % maxcitenames=2
    % maxbibnames=1,
    % minbibnames=3
]{biblatex} %Imports biblatex package
%Import the bibliography file(s)
%%%% bibliographic sources
\newcommand{\loadBibIfExists}[1]%%% load bib file only if it exists
{\IfFileExists{#1}
    {\addbibresource{#1}}%
    {}%
}
\isPortfolio
{
    \loadBibIfExists{../articles_common_files/biblatex_files/bibliography.bib}% FOR PORTFOLIO
}
{
    \isArticle{%%% make sure it really is an article, and not e.g. a standalone file
        \loadBibIfExists{../../articles_common_files/biblatex_files/bibliography.bib}% FOR SINGLE ARTICLES
    }{}
}


\DeclareBibliographyCategory{myMediationArticles}
\addtocategory{myMediationArticles}{\myarticleKey}


%%% Include all biblatex items in bibliography, even if not cited in the text
% \nocite{*}

%%% MY OWN TYPE AND RESPECTIVE FIELDS
%%%% My own references
\isPortfolio
{
  \loadBibIfExists{../articles_common_files/biblatex_files/myArticles.bib} % FOR PORTFOLIO
}
{
  \isArticle{%%% make sure it really is an article, and not e.g. a standalone file
    \loadBibIfExists{../../articles_common_files/biblatex_files/myArticles.bib} % FOR SINGLE ARTICLES
  }{}
}




\usepackage{filecontents}

\begin{filecontents}{mydatamodels.dbx}
  %%%%%%%%%%%%%%%%%%%%%%%%%%
  % CREATE DATAMODEL ENTRY TYPES (equivalent to the predefined "article", "book", etc)
  \DeclareDatamodelEntrytypes{myarticle}
  %
  %%%%%%%%%%%%%%%%%%%%%%%%%%
  % CREATE FIELDS OF DIFFERENT TYPES
  % literal: Used for text fields or data that should be taken as is, without special formatting or further interpretation. Fields of this type are used for plain text, such as names, titles, or descriptions.
  \DeclareDatamodelFields[type=field, datatype=literal]
  % Important to add a "%" percent symbol or "," comma after each field
  {
    subtitle,
    abstract,
    targetPublication,
    audienceLevel,
    wordMin,
    wordMax,
    charMin,
    charMax,
    copyright,
    mainSourceKey,
  }
  %
  % name: Designed specifically for names, like authors, editors, and translators. biblatex applies specific formatting rules (e.g., for initials, ordering) to these fields, which can be lists of names.
  \DeclareDatamodelFields[type=list,datatype=name]
  % Important to add a "%" percent symbol or "," comma after each field
  {
    illustrator,
    reviewer,
    translator,
    thank,
    discipline,
    % Finish every line with a comma
  }
  %
  % date: Used for fields containing dates, which enables biblatex to format them according to the bibliography style and locale. Dates are parsed, and users can specify exact or approximate dates (e.g., 2001, 2022-03-15).
  \DeclareDatamodelFields[type=field, datatype=date, skipout]
  % Important to add a "%" percent symbol or "," comma after each field
  {
    % Finish every line with a comma
  }
  %
  % verbatim: Used for data that should be reproduced exactly as written without additional formatting or escaping of special characters. This is helpful for URLs, DOIs, or other technical strings that must remain unchanged. "literal" is for text needing minimal but some formatting, while "verbatim" is for data that must remain entirely unchanged.
  \DeclareDatamodelFields[type=field, datatype=verbatim]
  % Important to add a "%" percent symbol or "," comma after each field
  {
    % Finish every line with a comma
  }
  %
  %%%%%%%%%%%%%%%%%%%%%%%%%%
  % ADD RELEVANT FIELDS HERE FROM DECLARATIONS ABOVE
  \DeclareDatamodelEntryfields[myarticle]
    % Important to add a "%" percent symbol or "," comma after each field
  {
    title,
    subtitle,
    illustrator,
    translator,
    reviewer,
    thank,
    abstract,
    targetPublication,
    audienceLevel,
    wordMin,
    wordMax,
    charMin,
    charMax,
    keywords,
    discipline,
    copyright,
    mainSourceKey,
    % Finish every line with a comma
    }
\end{filecontents}


%%% get around this formatting by using citefield directly
\DeclareFieldFormat[myarticle]{title}{%
    % \textcolor{myColorSecondary}{
        % \mkbibquote{% Surround in quotes
          #1\isdot
        % }%
    % }
}

\DeclareFieldFormat[myarticle]{illustrator}{
    \printtext{illustrator}
    \textcolor{red}{
        \mkbibquote{#1\isdot}
    }
}



\newbibmacro*{myArticleBibMacro}{%
  \printfield{title}%
  %
  % \newunit\newblock %  Creates a new unit ensuring that what's printed next starts in a new logical block.
  \ifnameundef{author}
  {}
  {
    \newunit\newblock %
    \printtext{Written by}
    \printnames{author}%
    \setunit{\addcomma\space}%
  }
  %
  %
  %
  \newunit\newblock %
  \ifnameundef{illustrator}
  {}
  {
    \printtext{Illustrated by}
    \printnames{illustrator}%
    \setunit{\addcomma\space}%
  }
  %
  %
  %
  \newunit\newblock %
  \ifnameundef{reviewer}
  {}
  {
    \printtext{Reviewed by}
    \printnames{reviewer}%
    \setunit{\addcomma\space}%
  }
  %
  %
  %
  \newunit\newblock %
  \iffieldundef{targetPublication}
  {}
  {
    \printtext{To be published in}
    \printfield{targetPublication}%
    \setunit{\addcomma\space}%
  }
  \newunitpunct\addperiod % Adds a period at the end
}


% Define a custom bibliography style
\DeclareBibliographyDriver{myarticle}{%
  \usebibmacro{bibindex}% Prints the bibliography index if indexing is enabled
  \usebibmacro{begentry}% Marks the beginning of the bibliography entry, setting up any required formatting
  \usebibmacro{myArticleBibMacro}%
  \usebibmacro{finentry}% Marks the end of the bibliography entry, completing any required formatting or spacing.
}


\newcommand{\setSurnameCommaN}{%%% Surname N.
  \namepartfamily\addcomma\addspace \namepartgiveni\addcomma\isdot%
}
\newcommand{\setNameSpaceSurname}{%%% Name Surname
  \namepartgiven\addspace\namepartfamily\isdot%
}

% CHOOSE NAME FORMAT HERE
\newcommand{\chooseNameFormat}{%
      %%% Surname, N.
      % \setSurnameCommaN%
      %%% Name Surname
      \setNameSpaceSurname%
}

\newboolean{myHighlight}
\newcommand{\setMyHighlight}[1]{%
  \begingroup%
      % \mkbibbold{%
      % \color{myColorSecondary}%
      #1%
      % }%
  \endgroup%
}%

\newcommand{\highlightName}[2]{%
  \DeclareNameFormat{#2}{%
  \setboolean{myHighlight}{false}%
    \renewcommand{\do}[1]{\expandafter\ifstrequal\expandafter{\namepartfamily}{####1}{\setboolean{myHighlight}{true}}{}}%
    \docsvlist{#1}%
    %%%%%········· FIRST ENTRY
    \ifthenelse{\value{listcount}=1}
    {%
      {\expandafter\ifthenelse{\boolean{myHighlight}}{\setMyHighlight{%
        %%% ENTRY
        \chooseNameFormat%
        }}{%
        %%% ENTRY
        \chooseNameFormat%
      }}%
      %%%%%········· MIDDLE ENTRIES
    }{\ifnumless{\value{listcount}}{\value{liststop}}
      {\expandafter\ifthenelse{\boolean{myHighlight}}{\setMyHighlight{%
        %%% ENTRY
        \addcomma\addspace\chooseNameFormat%
        }}{%
        %%% ENTRY
        \addcomma\addspace\chooseNameFormat%
      }}%
      %%%%%········· LAST ENTRY
      {\expandafter\ifthenelse{\boolean{myHighlight}}{and\addspace
      \setMyHighlight{%
        %%% ENTRY
        \chooseNameFormat%
      }}{%
        %%% ENTRY
        and\addspace\chooseNameFormat%
        }}%
      }
    \ifthenelse{\value{listcount}<\value{liststop}}
    {\addcomma\space}{}
  }
}

\highlightName{Surname, Name}{author}
\highlightName{Surname, Name}{illustrator}
\highlightName{Surname, Name}{reviewer}


\AtDataInput{\stepcounter{%
    totalCitationsAltogether%
    }} % Increment with each citation
% AtDataInput{} is triggered at the instant of each citation, whereas AtEveryBibitem{} counts only after Bib printing




% Command to add bibliography section
\newcommand{\addBibliography}{
    \isPortfolio{%
    \newcommand{\myCitationCounter}{1} %%% Print in ANY case since 1<>0
    }
    {
    \newcommand{\myCitationCounter}{%
        \totvalue%
        {totalCitationsInArticle:\myarticleKeyCore}%
        }
    }

    \ifthenelse{\boolean{isIncludeBiblio} \and \myCitationCounter>0}{
        \newpage
        \begin{SplitColumnsInTwo}%[true]
        \updateRibbons{\textbf{\TEXTbibliography}}{}
        \mySectionTitle{\TEXTbibliography}
        % This document contains \total{totalCitationsInArticle:\myarticleKeyCore}\ citation(s).
        \printbibliography[
            heading=none, % "bibintoc" adds the title to the table of contents. "none" to exclude.
            % title={My bibliography title} % Add title above bibliography
            % type=report,
            notcategory=myMediationArticles% Exclude my articles
            ]
        \end{SplitColumnsInTwo}
    }{}
}




\newcommand{\myCite}[1]{%
    \ifthenelse{\boolean{isIncludeCitations}}{% whether or not to include citations in article
    \stepcounter{%
    totalCitationsInArticle:\myarticleKeyCore% COUNTS REPEATS!
    }%
        \ifthenelse{\boolean{isIncludeCitationsInFootnotes}}{% whether to show citations in footnotes or not
            \footcite{#1}%
        }{%
            \cite{#1}%
        }%
    }{}
}



% raise inline text of different size to be aligned vertically
\newcommand*\raiseup[2]{%
        \begingroup%
        \setbox0\hbox{#1\strut #2}%
        \leavevmode%
        % Change formula to adjust height
        \raise\dimexpr (\ht\strutbox - \ht0)/3 \box0%
        \endgroup%
}

% Change missing reference message for Biblatex
\usepackage{xpatch}
\newcommand{\myUnknownRefSymbol}{????}
\makeatletter%
\def\abx@missing@entry#1{%
\raiseup{\tiny}{%
    \textcolor{myColorDanger}{%
        \abx@missing{[\myUnknownRefSymbol\ #1 \myUnknownRefSymbol]}%
    }%
    }%
}
\makeatother%


%%% Custom cite commands

% command to apply to prenotes and custom inputs
\newcommand{\genericPrenote}[1]
{\textcolor{myGrayMed}{#1\addcolon\space}}
% custom field format
\DeclareFieldFormat{myLabelFormat}{\genericPrenote{\titlecap{#1}}}
%
% field format for prenotes
\DeclareFieldFormat{prenote}
{\genericPrenote{#1}}
%
% custom empty entry format
\DeclareFieldFormat{myEntrymptyEntry}{\textcolor{red}{#1}}
%
% field format for DOI specifically (auto applies)
\DeclareFieldFormat{doi}{%
%   \mkbibacro{DOI} % prints label by default
  \ifhyperref
    {\href{https://doi.org/#1}{\nolinkurl{#1}}}
    {\nolinkurl{#1}}}
%
% field format for URL specifically (auto applies)
\DeclareFieldFormat{url}{%
% \mkbibacro{URL} % prints label by default
\url{#1}}
%
% field format for ISSN specifically (auto applies)
\DeclareFieldFormat{issn}{%
% \mkbibacro{ISSN} % prints label by default
#1}
% Define separator between citation and postnote
\renewcommand{\postnotedelim}{\space}

% \usepackage{natbib}
% \setcitestyle{comma}

% Macro to check if an entry is empty, and print something if TRUE or FALSE
\newcommand{\checkIfNoEntryFound}[2]{%
    \iffieldundef{\myEntry}% IS IT A FIELD ?
    {%
        \ifnameundef{\myEntry}% IS IT A NAME ?
        {%
            \iflistundef{\myEntry}% IS IT A LIST ?
            {#1}%
            {#2}%
        }%
        {#2}%
    }%
    {#2}%
}%
\newcommand{\checkIfNoEntryFoundConditional}[2]{%
    \ifthenelse{\boolean{isIncludeMissingBibEntries}}
    {%
    % \textcolor{myColorSuccess}{TRUE}
    #2%
    }%
    {%
        \checkIfNoEntryFound{#1}{#2}
    }%
}%


\DeclareCiteCommand{\myCiteCommand}%
    {% PRENOTE
    % \textcolor{orange}{\myEntry}
    \checkIfNoEntryFound{%
            % \textcolor{red}{Missing entry!}
        }{%
            \renewcommand{\genericPrenote}[1]{#1}% to remove formatting from prenote
            \usebibmacro{prenote}%
        }%
    }
    {
        \iffieldundef{\myEntry}% IS IT A FIELD ?
        {%
            \ifnameundef{\myEntry}% IS IT A NAME ?
            {%
                \iflistundef{\myEntry}% IS IT A LIST ?
                {%
                    % \ifthenelse{\boolean{isIncludeMissingBibEntries}}{%
                    \isDraftDebugger{
                        \printtext[myEmptyEntry]
                        {\na}
                        }{}%
                        % If it is none of the below
                    % }{}%
                }%
                {%
                    \printlist{\myEntry}% if it is a biblatex list
                }%
            }%
            {%
                \printnames{\myEntry}% if it is a biblatex name
            }%
        }%
        {%
            \printfield{\myEntry}% if it is a biblatex field
        }%
    }
    {}
    {% POSTNOTE
    \checkIfNoEntryFound{%
            % Missing entry!
        }{%
            \usebibmacro{postnote}%
        }%
    }

\DeclareCiteCommand{\myCiteWithLabelCommand}%
    {%
        \checkIfNoEntryFoundConditional{%
            % Missing entry!
        }{%
            \item[%
                \iffieldundef{prenote}%
                {%
                    \printtext[myLabelFormat]{\myEntry:}%
                    % \setunit{\prenotedelim}
                }%
                {%
                    \usebibmacro{prenote}%
                }%
            ]
        }%
    }%
    {%
        \iffieldundef{\myEntry}% IS IT A FIELD ?
        {%
            \ifnameundef{\myEntry}% IS IT A NAME ?
            {%
                \iflistundef{\myEntry}% IS IT A LIST ?
                {%
                    \ifthenelse{\boolean{isIncludeMissingBibEntries}}{%
                        \isDraftDebugger{
                            \printtext[myEmptyEntry]{\na}%
                            }{}%
                    }{}%
                }%
                {%
                    \printlist{\myEntry}% if it is a biblatex list
                }%
            }%
            {%
                \printnames{\myEntry}% if it is a biblatex name
            }%
        }%
        {%
            \printfield{\myEntry}% if it is a biblatex field
        }%
    }%
    {%
        \multicitedelim%
    }%
    {%
        \checkIfNoEntryFoundConditional{%
            % Missing entry!
        }{%
            \usebibmacro{postnote}%
        }%
    }%


% placeholder command to hold entry label
\newcommand{\myEntry}{}
% entrypoint commands for the citation that picks on the above cite command to generalize it
\NewDocumentCommand{\myCiteEntryWithLabel}
{
    m% #1 citation key
    m% #2 citation entry key
    O{}% #3 prenote
    O{}% #4 postnote
}{%
    \renewcommand{\myEntry}{#2}%
    \myCiteWithLabelCommand[#3][#4]{#1}%
}%

% entrypoint command for the citation that picks on the above cite command to generalize it
\NewDocumentCommand{\myCiteEntry}
{
    m% #1 citation key
    m% #2 citation entry key
    O{}% #3 prenote
    O{}% #4 postnote
}{%
    \renewcommand{\myEntry}{#2}%
    % \textcolor{red}{#2}
    \myCiteCommand[#3][#4]{#1}%
}%
