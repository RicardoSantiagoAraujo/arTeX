
\usepackage[
  acronym,
  section, % to avoid page break if placed after chapter heading
  nopostdot=true, % add/remove "." after each definition
  nogroupskip=false, % add/remove skip after each group
  nonumberlist % add/remove references to appearances by page number
]{glossaries}%import glossary as well as acronym features

% Change glossary/acronym nsame font
\renewcommand{\glsnamefont}[1]{\textsf{\textcolor{myColorPrimary}{#1}}}
% Change glossary/acronym page number style (if "nonumberlist" is off)
%\renewcommand{\glossaryentrynumbers}[1]{\textcolor{myColorPrimary}{#1}}
% Change glossary/acronym alphabetic navigation line style
\renewcommand*{\glslistnavigationitem}[1]
{\item \textcolor{myColorPrimary}{\textbf{#1}}}
% Change glossary/acronym alphabetic letter grouping style
\renewcommand*{\glslistgroupheaderfmt}[1]
{\textcolor{myColorPrimary}{\textbf{#1}}}

\makenoidxglossaries

%%% GLOSSARY STYLE
\newcommand{\glossaryStyle}{}
\ifthenelse{\boolean{isPrintVersion}}
{
  \renewcommand{\glossaryStyle}{listgroup}
}
{
  \renewcommand{\glossaryStyle}{listhypergroup}
}
\setglossarystyle{\glossaryStyle}
%%% Style options:
% list. Writes the defined term in boldface font
% altlist. Inserts newline after the term and indents the description.
% listgroup. Group the terms based on the first letter.
% listhypergroup. Adds hyperlinks at the top of the index.


\NewDocumentCommand{\glossaryItem}
{
  O{en} % LANGUAGE
  m % KEY
  O{} % NAME
  m % DESCRIPTION
  O{} % ABBREVIATION
}
{
    \ifthenelse{\equal{\myLanguage}{#1}}{
        \stepcounter{totalGlossaryEntriesAltogether} % increment counter

        %%% CREATE NEW GLOSSARY ENTRY
        % If #3 is empty, use #2 as name
        \ifthenelse{\equal{#3}{}}
        {% IF THEN
          \newglossaryentry{#2}
          {
              name={#2}, % Reuse entry
              description={\myGlossaryDescription{#2}{#4}{#5}}
          }
        }
        %%%%%%%%%%%%%%%%%%%%%%%%%%%%%%%%%%%%%%%%%%%%%%%%%%%%%%%%%%%%%%%%%%%%%%
        {% ELSE
          \newglossaryentry{#2}
          {
              name={#3},
              description={\myGlossaryDescription{#2}{#4}{#5}}
          }
        }
    }{}
}

\NewDocumentCommand{\myGlossaryDescription}{
  m % glossary key
  m % Description
  m % Abreviation key
}{%
\isDraftDebugger{[KEY: \detokenize{#1}\ ]}{}
#2 \includeAbrevInGlossary{#3}%
}

\NewDocumentCommand{\includeAbrevInGlossary}
{
  m % abbreviation key
}{%
  \ifthenelse{\equal{#1}{}}% check if empty
  {\isDraftDebugger{no abbreviation}{}}
  {Abbreviated as \acrshort{#1}.}
}



\NewDocumentCommand{\abreviationsItem}
{
  O{en} % LANGUAGE
  m % KEY
  O{} % NAME
  m % MEANING
}
{
    \ifthenelse{\equal{\myLanguage}{#1}}{
        \stepcounter{totalAbreviationsAltogether} % increment counter
        %%% CREATE NEW GLOSSARY ENTRY
        % If #3 is empty, use #2 as name
        \ifthenelse{\equal{#3}{}}
        {% IF THEN
        \newacronym{#2}{#2}{\isDraftDebugger{[abrev. key: #2]}{}[red]#4}
        }
        {% ELSE
        \newacronym{#2}{#3}{\isDraftDebugger{[abrev. key: #2]}{}[red]#4}
        }
    }{}
}

% FORMAT GLOSSARY/ABBREVIATION LINK IN TEXT
\ifthenelse{\boolean{isDraft}}
{ % DRAFT MODE
  \renewcommand*{\glstextformat}[1]%
  {%
    \texttt{%
    \colorbox{myColorSuccess}%
    {%
      \textcolor{myColorWarning}{#1}%
    }%
    }%
    }%
}
{ % NORMAL MODE
  \renewcommand*{\glstextformat}[1]
  {%
  \ifthenelse{
    \boolean{isHighlightGlossaryAndAbreviations}
    % \AND
    % \(
    % \boolean{isIncludeGlossary}
    % \OR
    % \boolean{isIncludeAbreviations}
    % \)
    %
    }%
  {%
    \textbf{%
      \textcolor{myGrayDark}
      {%
        #1%
      }%
      }%
    }%
  }{}%
}





\newcommand{\addAbreviations}{

  \ifthenelse{
    \boolean{isIncludeAbreviations}
    \AND
      \(
      \boolean{isPrintUnusedAbreviations}
      \OR
      \totvalue{totalAbreviationsInArticle:\myarticleKeyCore}>0
      \)
    \AND
    \totvalue{totalAbreviationsAltogether}>0
    }{
    \newpage
    \begin{SplitColumnsInTwo}%[true]
      \updateRibbons{\textbf{Abreviations}}{}
      {
        \singlespacing
        \mySectionTitle{Abreviations and Acronyms}
        % Print Abreviations
        \printnoidxglossary[
          type=\acronymtype,
          title={},
          % toctitle=List of terms
          ]
        % Add even unused entries
        \ifthenelse{\boolean{isPrintUnusedAbreviations}}{
          \glsaddallunused[\acronymtype]
        }{}
      }
    \end{SplitColumnsInTwo}
  }
  {}
}



\newcommand{\addGlossary}{
  \ifthenelse{
    \boolean{isIncludeGlossary}
    \AND
      \(
      \boolean{isPrintUnusedGlossary}
      \OR
      \totvalue{totalGlossaryEntriesInArticle:\myarticleKeyCore}>0
      \)
    \AND
    \totvalue{totalGlossaryEntriesAltogether}>0
  }{
    \newpage
    \begin{SplitColumnsInTwo}%[true]
      \updateRibbons{\textbf{Glossary}}{}
      {
        \singlespacing
        \mySectionTitle{Glossary}
        % Print glossary
        \printnoidxglossary[
          type=main,
          title={},
          %  toctitle=List of terms
          ]
          % Add even unused entries
          \ifthenelse{\boolean{isPrintUnusedGlossary}}{
            \glsaddallunused[main]
          }{}
      }
    \end{SplitColumnsInTwo}
  }
  {}
}



\newcommand{\addAbreviationsPORTFOLIO}{
  \ifthenelse{\boolean{isIncludeAbreviations} \and 1>0}{
    % \newpage
    \cleardoublepage
    \begin{SplitColumnsInTwo}
    \myPortfolioChapter{Abreviations}
    {
      \phantomsection
      \singlespacing
      %%%
      % Print Abreviations
      \renewcommand*{\glsclearpage}{} % change default break behavior before glossary
      \printnoidxglossary[
        type=\acronymtype,
        % style=listgroup,
        title={}
        % toctitle={}% Dont comment out
      ]
      % Add even unused entries
      \ifthenelse{\boolean{isPrintUnusedAbreviations}}{
        \glsaddallunused[\acronymtype,]
      }{}
    }
    \end{SplitColumnsInTwo}
  }
  {}
}


\newcommand{\addGlossaryPORTFOLIO}{
  \ifthenelse{\boolean{isIncludeGlossary} \and 1>0}{
    % \newpage
    \cleardoublepage
    \begin{SplitColumnsInTwo}
    \myPortfolioChapter{Glossary}
    {%
      \phantomsection
      \singlespacing
      %%%
      % Print glossary
      \renewcommand*{\glsclearpage}{} % change default break behavior before glossary
      \printnoidxglossary[
        type=main,
        % style=listgroup,
        title={}
        % toctitle={}% Dont comment out
      ]
      % Add even unused entries
      \ifthenelse{\boolean{isPrintUnusedGlossary}}{
        \glsaddallunused[main]
      }{}
    }
    \end{SplitColumnsInTwo}
  }
  {}
}

\NewDocumentCommand{\myGLS}%
{%
  m% glossary key
  O{}% optional text to be used instead of name from glossary
}{%
  {%
    \ifthenelse{{\equal{\myarticleKeyCore}{}}}% check if key is still empty
    {}% if so, do nothing, since it means the article has not yet started
    {% else, step counter
      \stepcounter{totalGlossaryEntriesInArticle:\myarticleKeyCore}%
    }%
    \ifthenelse{\boolean{isIncludeGlossary}}%
    {%
      %GLOSSARY INCLUDED
      \isDraftDebugger{[gloss. key: \detokenize{#1}]}%
      {%
        \ifthenelse{%
          \equal{#2}{}%
          }%
          {\gls{#1}}% if no optional text was provided
          {\glslink{#1}{#2}}% if optional text was provided
      }%
    }%
    {%
      %GLOSSARY NOT INCLUDED
      \isDraftDebugger{[gloss. key: \detokenize{#1}]}%
      {%
        \ifthenelse{%
          \equal{#2}{}%
          }%
          {\gls*{#1}}% if no optional text was provided
          {\glslink*{#1}{#2}}% if optional text was provided
      }%
    }%
  }%
}

\NewDocumentCommand{\myAbrev}%
{%
  m% abreviation key
  O{}% optional text to be used instead of name from abreviation
}{%
  {%
    \ifthenelse{{\equal{\myarticleKeyCore}{}}}% check if key is still empty
    {}% if so, do nothing, since it means the article has not yet started
    {% else, step  counter
      \stepcounter{totalAbreviationsInArticle:\myarticleKeyCore}%
    }%
    \ifthenelse{\boolean{isIncludeAbreviations}}%
    {%
      %GLOSSARY INCLUDED
      \isDraftDebugger{[abrev. key: \detokenize{#1}]}%
      {%
        \ifthenelse{%
          \equal{#2}{}%
          }%
          {\gls{#1}}% if no optional text was provided
          {\glslink{#1}{#2}}% if optional text was provided
      }%
    }%
    {%
      %GLOSSARY NOT INCLUDED
      \isDraftDebugger{[abrev. key: \detokenize{#1}]}%
      {%
        \ifthenelse{%
          \equal{#2}{}%
          }%
          {\gls*{#1}}% if no optional text was provided
          {\glslink*{#1}{#2}}% if optional text was provided
      }%
    }%
  }%
}
